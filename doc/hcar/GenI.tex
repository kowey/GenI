\begin{hcarentry}[updated]{GenI}
\label{geni}
\report{Eric Kow}%05/08
\makeheader

GenI is a surface realiser for Tree Adjoining Grammars. Surface
realisation can be seen as the last stage in a natural language
generation pipeline. GenI in particular takes an FB-LTAG grammar and an
input semantics (a conjunction of first order terms), and produces the
set of sentences associated to the input semantics by the grammar.  It
features a surface realisation library, several optimisations, batch
generation mode, and a graphical debugger written in wxHaskell.  It was
developed within the TALARIS project and is free software licensed under
the GNU GPL.

GenI is available on Hackage, and can be installed via cabal-install.
We also have a mailing list at
\url{http://websympa.loria.fr/wwsympa/info/geni-users}.

\FurtherReading
\begin{compactitem}
\item \url{http://trac.loria.fr/~geni}
\item Paper from Haskell Workshop 2006:

\url{http://hal.inria.fr/inria-00088787/en}
\end{compactitem}
\end{hcarentry}
