\begin{hcarentry}[section]{GenI}
\label{geni}
\report{Eric Kow}
\status{unchanged, pinged}% done 2007-12-20, remove 05/2008
\makeheader

GenI is a surface realiser for Tree Adjoining Grammars. Surface
realisation can be seen as the last stage in a natural language
generation pipeline. GenI in particular takes an FB-LTAG grammar and an
input semantics (a conjunction of first order terms), and produces the
set of sentences associated to the input semantics by the grammar.  It
features a surface realisation library, several optimisations, batch
generation mode and a graphical debugger written in wxHaskell.  It was
developed within the TALARIS project and is free software licensed under
the GNU GPL.

\FurtherReading
\begin{compactitem}
\item \url{http://trac.loria.fr/~geni}
\item Paper from Haskell Workshop 2006:

\url{http://hal.inria.fr/inria-00088787/en}
\end{compactitem}
\end{hcarentry}
